\documentclass[12pt,a4paper]{article}
\usepackage[legalpaper, portrait, lmargin=1cm, rmargin=1cm, tmargin=2cm, bmargin=2cm]{geometry}
\usepackage{fancyhdr}
\usepackage{amsmath}
\usepackage{amssymb}
\usepackage{graphicx}
\usepackage{wrapfig}
\usepackage{blindtext}
\usepackage{hyperref}
\usepackage{pdflscape}
\usepackage{svg}
\usepackage{xcolor}
\usepackage[portuguese]{babel}

\graphicspath{ {./} }

\definecolor{linkcolor}{HTML}{1588e0}
\hypersetup{
  colorlinks=true,
  allcolors=linkcolor,
  pdftitle={Relatório Entrega 2 AMS 2023/2024 LEIC-A},
  pdfpagemode=FullScreen,
}

\pagestyle{fancy}
\fancyhf{}
\rhead{Grupo \textbf{50}}
\lhead{Relatório Entrega 2 AMS 2023/2024 LEIC-A}
\cfoot{Gonçalo Bárias (103124), Miguel Costa (103969) e Raquel Braunschweig (102624)}

\renewcommand{\footrulewidth}{0.2pt}
\renewcommand{\labelitemii}{$\circ$}
\renewcommand{\labelitemiii}{$\diamond$}

\begin{document}
\begin{titlepage}
  \begin{center}
    \vspace*{5cm}
    \Huge
    \textbf{Projeto AMS - Entrega 2}

    \vspace{0.5cm}
    \LARGE
    Grupo 50 | Turno L10 | LEIC-A

    \vspace{0.5cm}
    \large
    Prof. Sérgio Luís Proença Duarte Guerreiro

    \vfill
    \large
    \begin{minipage}{0.8\textwidth}
      \begin{itemize}
        \item[] \textbf{Gonçalo Bárias} (103124) - 14h
        \item[] \textbf{Miguel Costa} (103969) - 14h
        \item[] \textbf{Raquel Braunschweig} (102624) - 14h
      \end{itemize}
    \end{minipage}
  \end{center}
\end{titlepage}

\begin{landscape}
  \begin{figure}
    \centering
    \includesvg[inkscapelatex=false,width=1.6\textwidth]{assets/a1-fixed.svg}
    \caption{Diagrama do Contexto do Negócio}
    \label{fig:a1-fixed}
  \end{figure}
\end{landscape}

\begin{landscape}
  \begin{figure}
    \centering
    \includesvg[inkscapelatex=false,width=1.5\textwidth]{assets/a2-fixed.svg}
    \caption{Diagrama de Vista Geral do Negócio}
    \label{fig:a2-fixed}
  \end{figure}
\end{landscape}

\begin{landscape}
  \begin{figure}
    \centering
    \includesvg[inkscapelatex=false,width=1.6\textwidth]{assets/p1-fixed.svg}
    \caption{Diagrama do Processo de Gestão dos Fornecedores}
    \label{fig:p1-fixed}
  \end{figure}
\end{landscape}

\begin{landscape}
  \begin{figure}
    \centering
    \includesvg[inkscapelatex=false,width=1.6\textwidth]{assets/p2-fixed.svg}
    \caption{Diagrama do Processo de Gestão de Artigos}
    \label{fig:p2-fixed}
  \end{figure}
\end{landscape}

\begin{landscape}
  \begin{figure}
    \centering
    \includesvg[inkscapelatex=false,width=1.5\textwidth]{assets/uml1.svg}
    \caption{Diagrama de casos de uso (em UML) da aplicação SCM e respetivos atores}
    \label{fig:uml1}
  \end{figure}
\end{landscape}

\begin{landscape}
  \begin{figure}
    \centering
    \includesvg[inkscapelatex=false,width=1.45\textwidth]{assets/uml2.svg}
    \caption{Diagrama de classes (em UML) representando o modelo de domínio da aplicação SCM}
    \label{fig:uml2}
  \end{figure}
\end{landscape}

\begin{landscape}
  \begin{figure}
    \centering
    \includesvg[inkscapelatex=false,width=1.45\textwidth]{assets/uml3.svg}
    \caption{Diagrama de Máquina de Estados (em UML) do comportamento dos objetos da classe Fornecedor}
    \label{fig:uml3}
  \end{figure}
\end{landscape}

\begin{landscape}
  \begin{figure}
    \centering
    \includesvg[inkscapelatex=false,width=1.4\textwidth]{assets/sysml1.svg}
    \caption{Diagrama de casos de uso (em SysML) de um PDA e respetivos atores}
    \label{fig:sysml1}
  \end{figure}
\end{landscape}

\begin{landscape}
  \begin{figure}
    \centering
    \includesvg[inkscapelatex=false,width=1.45\textwidth]{assets/sysml2.svg}
    \caption{Diagrama de blocos (em SysML) de um PDA}
    \label{fig:sysml2}
  \end{figure}
\end{landscape}

\begin{landscape}
  \begin{figure}
    \centering
    \includesvg[inkscapelatex=false,width=1.4\textwidth]{assets/sysml3.svg}
    \caption{Diagrama interno de blocos (em SysML) de um PDA}
    \label{fig:sysml3}
  \end{figure}
\end{landscape}

\end{document}
